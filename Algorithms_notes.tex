\documentclass{myclass}
\title{DAA,DSA-Attempted self study}
\author{vk}
\newtheorem*{definition}{Definition}
\newtheorem*{claim}{Claim}
\date{}
\begin{document}

\maketitle

\hrule
\vspace{-0.15cm}
\begin{flushleft}
    14th Dec 2023
\end{flushleft}
\vspace{-0.15cm}
\hrule

\section*{Egyptian fractions}

\begin{definition}
    A rational number which in reduced form has a numerator of 1 is called as an egyptian fraction.
\end{definition}

\bld{FACT} Ancient Egyptians who also worked with base 10 number systems, actually had in thier syntax only these type of fractions
and all others were built from these basic fractions.
\\\\
\bld{Question:} Can egyptians actually write all fractions less than one in this way?
\\\\
\bld{Answer:} Yes as the following algortithm to obtain such an expansion shows.

\begin{algorithm}
    \caption*{Egyptifier($p/q\in \bb{Q}$)}
    \begin{algorithmic}[1] 
        \Ensure{$p<q \text{\bld{ and }} q\neq 0$}
        \State nume \ass p
        \State ret\_list \ass $[\,\,]$\hspace{0.5cm}\Comment This list will have all the denominators of the component egyptian fractions.
        \While {nume $\neq$ 0} 
            \State n \ass $\lceil q/p \rceil$
            \State ret\_list.append(n)
            \State nume \ass pn-q
        \EndWhile\\
        \bld{Output} ret\_list
    \end{algorithmic}
\end{algorithm}

\begin{claim}
    This algorithm works!
\end{claim}
\begin{prf}
    The idea is to use the monovariant \italic{nume}. On each iteration, by the choice of n, nume monotonously decreases 
    while being non-negative. This is so because,
    $$n=\lceil q/p \rceil \iff n-1<q/p\leq n \iff 1/n\leq p/q <1/(n-1) \iff 0 \leq (pn-q)/q \text{ and }pn-q<p$$
    \qed
\end{prf}

\hrule
\vspace{-0.15cm}
\begin{flushleft}
    15th Dec 2023
\end{flushleft}
\vspace{-0.15cm}
\hrule

\section*{Interval Scheduling}

\bld{Input} Certain events given as a list of 2-tuples $(s,f)$ denoting
the start and finish timings of the event. 
\\\\
\bld{Output} A sub-list of events from the original list ,that are non overlapping
and have the maximum possible number of events.
\\\\
This can naturally be usefull in several settings of practical interests like
picking the maximum number of rides in a theme park.

\newpage

The idea is the following. Keep picking the event that ends at the earliest, and does not clash with the ones already 
chosen. This greedy approach is implemeted in the following way. We use the heap data structure with the following terminology.

\begin{enumerate}
    \item \bld{MinHeapify} with some order function does heapify where the comaprision is made with the given function.
    \item \bld{Pop} removes that element from the heap.
    \item \bld{Peek} Returns the first element from the heap.
\end{enumerate}

\begin{algorithm}
    \caption*{Scheduler(L:list of event tuples)}
    \begin{algorithmic}[1]
        \State S \ass MinHeapify(L) with start time ordering.
        \State F \ass MinHeapify(L) with end time ordering.
        \State Ret\_list \ass $[\,\,]$
        \While{F \bld{not empty}}
            \State e \ass Peek(F)
            \State s \ass Peek(S)
            \While{$s[0]\leq e[1]$}
                \State S.Pop(s)
                \State F.Pop(s)
                \State s \ass Peek(S)
            \EndWhile
            \State Ret\_list.append(e)
        \EndWhile\\
        \bld{Output} Ret\_list
    \end{algorithmic}
\end{algorithm}

\begin{claim}
    This algorithm does return a list with maximum number of non-clashing(here after referred as compatible)
    events.
\end{claim}

\begin{prf}
    Say $\cc{O}$ is a list of events with $m$ elements such that it has the maximal number of compatible events.
    Let $R$ be the returned list. We shall prove that $|R|=|\cc{O}|$.\\
    The idea is that the greedy algorithm always stays ahed of this optimal list.

    \begin{claim}
        $R[i][1]\leq\cc{O}[i][1]$
    \end{claim}
    \begin{subprf}
        We prove by induction. For the first step its clear by design. We assume the statement holds for $j-1$. 
        But if so, then in the $jth$ step, $\cc{O}[j]$ begins after $\cc{R}[j-1]$ ends. Since the algorithm chooses the event with least finish
        time that does not clash with the existing ones, the above statement follows.
        \qed
    \end{subprf}

    Consequently, If the list $\cc{O}$ has got some $m$ events, then $R$ has atleast those many events.\qed
\end{prf}













\end{document}