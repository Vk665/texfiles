\documentclass{myclass}
\title{DAA,DSA-Attempted self study}
\author{vk}
\newtheorem*{definition}{Definition}
\newtheorem*{claim}{Claim}
\date{}
\begin{document}

\maketitle

\hrule
\vspace{-0.15cm}
\begin{flushleft}
    14th Dec 2023
\end{flushleft}
\vspace{-0.15cm}
\hrule

\section*{Egyptian fractions}

\begin{definition}
    A rational number which in reduced form has a numerator of 1 is called as an egyptian fraction.
\end{definition}

\bld{FACT} Ancient Egyptians who also worked with base 10 number systems, actually had in thier syntax only these type of fractions
and all others were built from these basic fractions.
\\\\
\bld{Question:} Can egyptians actually write all fractions less than one in this way?
\\\\
\bld{Answer:} Yes as the following algortithm to obtain such an expansion shows.

\begin{algorithm}
    \caption*{Egyptifier($p/q\in \bb{Q}$)}
    \begin{algorithmic}[1] 
        \Ensure{$p<q \text{\bld{ and }} q\neq 0$}
        \State nume \ass p
        \State ret\_list \ass $[\,\,]$\hspace{0.5cm}\Comment This list will have all the denominators of the component egyptian fractions.
        \While {nume $\neq$ 0} 
            \State n \ass $\lceil q/p \rceil$
            \State ret\_list.append(n)
            \State nume \ass pn-q
        \EndWhile\\
        \bld{Output} ret\_list
    \end{algorithmic}
\end{algorithm}

\begin{claim}
    This algorithm works!
\end{claim}
\begin{prf}
    The idea is to use the monovariant \italic{nume}. On each iteration, by the choice of n, nume monotonously decreases 
    while being non-negative. This is so because,
    $$n=\lceil q/p \rceil \iff n-1<q/p\leq n \iff 1/n\leq p/q <1/(n-1) \iff 0 \leq (pn-q)/q \text{ and }pn-q<p$$
    \qed
\end{prf}

\hrule
\vspace{-0.15cm}
\begin{flushleft}
    15th Dec 2023
\end{flushleft}
\vspace{-0.15cm}
\hrule

\section*{Interval Scheduling}

\bld{Input} Certain events given as a list of 2-tuples $(s,f)$ denoting
the start and finish timings of the event. 
\\\\
\bld{Output} A sub-list of events from the original list ,that are non overlapping
and have the maximum possible number of events.
\\\\
This can naturally be usefull in several settings of practical interests like
picking the maximum number of rides in a theme park.

\newpage

The idea is the following. Keep picking the event that ends at the earliest, and does not clash with the ones already 
chosen. This greedy approach is implemeted in the following way. We use the heap data structure with the following terminology.

\begin{enumerate}
    \item \bld{MinHeapify} with some order function does heapify where the comaprision is made with the given function.
    \item \bld{Pop} removes that element from the heap.
    \item \bld{Peek} Returns the first element from the heap.
\end{enumerate}

\begin{algorithm}
    \caption*{Scheduler(L:list of event tuples)}
    \begin{algorithmic}[1]
        \State S \ass MinHeapify(L) with start time ordering.
        \State F \ass MinHeapify(L) with end time ordering.
        \State Ret\_list \ass $[\,\,]$
        \While{F \bld{not empty}}
            \State e \ass Peek(F)
            \State s \ass Peek(S)
            \While{$s[0]\leq e[1]$}
                \State S.Pop(s)
                \State F.Pop(s)
                \State s \ass Peek(S)
            \EndWhile
            \State Ret\_list.append(e)
        \EndWhile\\
        \bld{Output} Ret\_list
    \end{algorithmic}
\end{algorithm}

\begin{claim}
    This algorithm does return a list with maximum number of non-clashing(here after referred as compatible)
    events.
\end{claim}

\begin{prf}
    Say $\cc{O}$ is a list of events with $m$ elements such that it has the maximal number of compatible events.
    Let $R$ be the returned list. We shall prove that $|R|=|\cc{O}|$.\\
    The idea is that the greedy algorithm always stays ahed of this optimal list.

    \begin{claim}
        $R[i][1]\leq\cc{O}[i][1]$
    \end{claim}
    \begin{subprf}
        We prove by induction. For the first step its clear by design. We assume the statement holds for $j-1$. 
        But if so, then in the $jth$ step, $\cc{O}[j]$ begins after $\cc{R}[j-1]$ ends. Since the algorithm chooses the event with least finish
        time that does not clash with the existing ones, the above statement follows.
        \qed
    \end{subprf}

    Consequently, If the list $\cc{O}$ has got some $m$ events, then $R$ has atleast those many events.\qed
\end{prf}

\hrule
\vspace{-0.15cm}
\begin{flushleft}
    16th Dec 2023
\end{flushleft}
\vspace{-0.15cm}
\hrule

\section*{Weighted Interval Scheduling}

Not all rides in a theme park are equally liked. Some are better than others. Say this is denoted by their weight.
How do we pick a compatible set of rides with the largest wieght?

\bld{Input} List of tuples of events and corresponding weights.
\\\\
\bld{Output} Sub-list of the input such that the events are all compatible and 
add to maximal weight.
\\\\
The idea is again simple. We shall build the set slowly such that at all points 
the list has the heaviest-compatible list upto a certain finish time. When the finish time of a new event is encountered, the weight of having it compared with the
weight of not having it. Correspondingly the list is updated.
\\\\
The procedure can be described as follows.

\begin{enumerate}
    \item Sort the evests with start times and finish times.
    \item Pick the event that ends the earliest.
    \item Move to the event that ends second earliest.
    \item If no clash, add it. If clashed, compare and choose.
    \item At an arbitrary state, say the max finish time of chosen events is f. Move to the event that ends after but closest to f. 
    \item If no clash add it. If clashed, compare having it and not having it, based on which wether to choose it.
    \item Go through all events.
\end{enumerate}

Here is a pseudocode.

\begin{algorithm}
    \caption*{Weighted-scheduler(E:list of 3-tuples)}
    \begin{algorithmic}[1]
        \State F \ass Sort(E) with finish key.
        \State Ret\_list \ass $[\,\,]$
        \State Part-sums \ass $[\,\,]$
        \Procedure{Find-Clash}{L:list of compatible events,e:some event}
            \State Binary-Search to return the index of the least element that clashes with e in L
        \EndProcedure
        \For{e \bld{in} F}
            \State j \ass Find-Clash(Ret\_list,e) 
            \If{Part-sums[j-1]+e[3]$>$Part-sums[-1]}
                \State Ret\_list \ass Ret\_list[0:j-1].append(e)
                \State Part-sums \ass Part-sums[0:j-1].append(Part-sums[j-1]+e[3])
            \EndIf
        \EndFor\\
        \bld{Output} Ret\_list
    \end{algorithmic}
\end{algorithm}

\hrule
\vspace{-0.15cm}
\begin{flushleft}
    17th Dec 2023
\end{flushleft}
\vspace{-0.15cm}
\hrule

\section*{Coins problem and Dynamic programming}

\bld{Question} Say there are certain denominations of coins, and you have to give a certain 
total amount of money to someone. What is the least amount of coins needed?
\\\\
Ofcourse this is a problem of high interest. Given a set of denominations $\cc{D}$ ans a target money T,
let $M(n,\cc{D})$ denote our answer, A smallest list of coins of total T. Then our solution is as follows.
$$M(n,\cc{D}) = \text{argmin }len\{(M(n-d,\cc{D})).\text{append}(d)|d\in\cc{D}\}$$
Now this basically reduces the problem of solving a bigger problem into solving a lot of subproblems.
So we essentially do not want to solve the subroblems repeatedly. Hence we need a register of the subproblems that have been solved.
Such an approach to breaking down problems is called as dynamic programming. First we shall write the brute force algortihm.

\vspace{-0.1cm}
\begin{algorithm}
    \caption*{Brute-force(T:total,$\cc{D}$:List of denominations)}
    \begin{algorithmic}[1]
        \If{$T\leq 0$}
            \State \bld{Output} $[]$
        \EndIf
        \State Ret\_list \ass $[]$
        \Procedure{Convinient-min}{$a,b$}
            \If{a \italic{isempty}}
                \State \bld{Output} b
            \ElsIf{b \italic{isempty}}
                \State \bld{Output} a
            \Else 
                \State \bld{Output} lesser length list of the two.
            \EndIf
        \EndProcedure
        \For {$d\in\cc{D}$}
            \State temp \ass Brute-force($T-d,\cc{D}$)
            \State Ret\_list \ass Convinient-min(temp,Ret\_list)
        \EndFor
        \State \bld{Output} Ret\_list
    \end{algorithmic}
\end{algorithm}

\newpage

Here it is easily seen that the same sub-problem is solved many times in different 
calls of the function. We can greatly save this time by just using memory. Consider the modified code below.
Here the curly brace $\{\,\,\,\}$ denotes an empty dictionary. (The code in in the next page)
This is code is many times fater because it only calls the function reursively $\cc{O}(n)$ time,
which is much better, than the exponentially large number of recursive calls made otherwise.
\\
We can recast the same idea in a bottom up manner. We see that we reduce the problem to that sub-problem that has the most optimal output.
We shall call that sub-problem its prdecessor. Here is the implementation.
\begin{algorithm}
    \caption*{Bottom-up-formulation(T:total,$\cc{D}$:List of denominations)}
    \begin{algorithmic}[1]
        \State Pre \ass $arr[T+1]$   \Comment{array of length T+1}
        \State Number \ass $arr[T+1]$   \Comment{array of length T+1}
        \State Pre[0],Number[0],i \ass 0,0,1
        \While{$i \leq T$}
            \State Temp \ass $\infty$
            \For{$d\in\cc{D}$ \bld{and} $d \leq i$}
                \If{Temp>Number[i-d]} 
                    \State Pre[i] \ass i-d 
                    \State Number[i] \ass Number[i-d]+1
                \EndIf
            \EndFor
        \State i \ass i+1
        \EndWhile
        \State \bld{Output} Number[T+1]
    \end{algorithmic}
\end{algorithm}

The optimal sequence of coins can be inferred from the predecessor list. Follow the predecessor of T until you hit zero. The
succesive differences of the generated sequence is the list of coins.

\newpage
\begin{algorithm}
    \caption*{Brute-force-withmemory(T:total,$\cc{D}$:List of denominations)}
    \begin{algorithmic}[1]
        \If{$T\leq 0$}
            \State \bld{Output} $[]$
        \EndIf
        \State memo \ass $\{\,\,\,\}$   \Comment{Entries stored as $T:Min(T,\cc{D})$}
        \If{T \bld{in} memo}
            \State \bld{Output} memo[T]
        \EndIf
        \State Ret\_list \ass $[]$
        \Procedure{Convinient-min}{$a,b$}
            \If{a \italic{isempty}}
                \State \bld{Output} b
            \ElsIf{b \italic{isempty}}
                \State \bld{Output} a
            \Else 
                \State \bld{Output} lesser length list of the two.
            \EndIf
        \EndProcedure
        \For {$d\in\cc{D}$}
            \State temp \ass Brute-force($T-d,\cc{D}$)
            \State Ret\_list \ass Convinient-min(temp,Ret\_list)
        \EndFor
        \State memo.add(T:Ret\_list)
    \end{algorithmic}
\end{algorithm}

The correctness of these procedures is guarenteed by the recursive relation that they solve.

\section*{Dictionary DS}

Here we shall see an implementation of the \bld{Dictionary} data structure. It supports the 
following operations.

\begin{itemize}
    \item \bld{INSERT(val,key)} procedure puts a certain value with key into the dictionary.
    \item \bld{SEARCH(key)} procedure outputs the value of the given key.
    \item \bld{DELETE(key)} procedure removes the value associated with a certain key.
\end{itemize}

An implementation of it using binary search trees.




\end{document}